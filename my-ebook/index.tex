% Options for packages loaded elsewhere
% Options for packages loaded elsewhere
\PassOptionsToPackage{unicode}{hyperref}
\PassOptionsToPackage{hyphens}{url}
\PassOptionsToPackage{dvipsnames,svgnames,x11names}{xcolor}
%
\documentclass[
  letterpaper,
  DIV=11,
  numbers=noendperiod]{scrreprt}
\usepackage{xcolor}
\usepackage{amsmath,amssymb}
\setcounter{secnumdepth}{5}
\usepackage{iftex}
\ifPDFTeX
  \usepackage[T1]{fontenc}
  \usepackage[utf8]{inputenc}
  \usepackage{textcomp} % provide euro and other symbols
\else % if luatex or xetex
  \usepackage{unicode-math} % this also loads fontspec
  \defaultfontfeatures{Scale=MatchLowercase}
  \defaultfontfeatures[\rmfamily]{Ligatures=TeX,Scale=1}
\fi
\usepackage{lmodern}
\ifPDFTeX\else
  % xetex/luatex font selection
\fi
% Use upquote if available, for straight quotes in verbatim environments
\IfFileExists{upquote.sty}{\usepackage{upquote}}{}
\IfFileExists{microtype.sty}{% use microtype if available
  \usepackage[]{microtype}
  \UseMicrotypeSet[protrusion]{basicmath} % disable protrusion for tt fonts
}{}
\makeatletter
\@ifundefined{KOMAClassName}{% if non-KOMA class
  \IfFileExists{parskip.sty}{%
    \usepackage{parskip}
  }{% else
    \setlength{\parindent}{0pt}
    \setlength{\parskip}{6pt plus 2pt minus 1pt}}
}{% if KOMA class
  \KOMAoptions{parskip=half}}
\makeatother
% Make \paragraph and \subparagraph free-standing
\makeatletter
\ifx\paragraph\undefined\else
  \let\oldparagraph\paragraph
  \renewcommand{\paragraph}{
    \@ifstar
      \xxxParagraphStar
      \xxxParagraphNoStar
  }
  \newcommand{\xxxParagraphStar}[1]{\oldparagraph*{#1}\mbox{}}
  \newcommand{\xxxParagraphNoStar}[1]{\oldparagraph{#1}\mbox{}}
\fi
\ifx\subparagraph\undefined\else
  \let\oldsubparagraph\subparagraph
  \renewcommand{\subparagraph}{
    \@ifstar
      \xxxSubParagraphStar
      \xxxSubParagraphNoStar
  }
  \newcommand{\xxxSubParagraphStar}[1]{\oldsubparagraph*{#1}\mbox{}}
  \newcommand{\xxxSubParagraphNoStar}[1]{\oldsubparagraph{#1}\mbox{}}
\fi
\makeatother

\usepackage{color}
\usepackage{fancyvrb}
\newcommand{\VerbBar}{|}
\newcommand{\VERB}{\Verb[commandchars=\\\{\}]}
\DefineVerbatimEnvironment{Highlighting}{Verbatim}{commandchars=\\\{\}}
% Add ',fontsize=\small' for more characters per line
\usepackage{framed}
\definecolor{shadecolor}{RGB}{241,243,245}
\newenvironment{Shaded}{\begin{snugshade}}{\end{snugshade}}
\newcommand{\AlertTok}[1]{\textcolor[rgb]{0.68,0.00,0.00}{#1}}
\newcommand{\AnnotationTok}[1]{\textcolor[rgb]{0.37,0.37,0.37}{#1}}
\newcommand{\AttributeTok}[1]{\textcolor[rgb]{0.40,0.45,0.13}{#1}}
\newcommand{\BaseNTok}[1]{\textcolor[rgb]{0.68,0.00,0.00}{#1}}
\newcommand{\BuiltInTok}[1]{\textcolor[rgb]{0.00,0.23,0.31}{#1}}
\newcommand{\CharTok}[1]{\textcolor[rgb]{0.13,0.47,0.30}{#1}}
\newcommand{\CommentTok}[1]{\textcolor[rgb]{0.37,0.37,0.37}{#1}}
\newcommand{\CommentVarTok}[1]{\textcolor[rgb]{0.37,0.37,0.37}{\textit{#1}}}
\newcommand{\ConstantTok}[1]{\textcolor[rgb]{0.56,0.35,0.01}{#1}}
\newcommand{\ControlFlowTok}[1]{\textcolor[rgb]{0.00,0.23,0.31}{\textbf{#1}}}
\newcommand{\DataTypeTok}[1]{\textcolor[rgb]{0.68,0.00,0.00}{#1}}
\newcommand{\DecValTok}[1]{\textcolor[rgb]{0.68,0.00,0.00}{#1}}
\newcommand{\DocumentationTok}[1]{\textcolor[rgb]{0.37,0.37,0.37}{\textit{#1}}}
\newcommand{\ErrorTok}[1]{\textcolor[rgb]{0.68,0.00,0.00}{#1}}
\newcommand{\ExtensionTok}[1]{\textcolor[rgb]{0.00,0.23,0.31}{#1}}
\newcommand{\FloatTok}[1]{\textcolor[rgb]{0.68,0.00,0.00}{#1}}
\newcommand{\FunctionTok}[1]{\textcolor[rgb]{0.28,0.35,0.67}{#1}}
\newcommand{\ImportTok}[1]{\textcolor[rgb]{0.00,0.46,0.62}{#1}}
\newcommand{\InformationTok}[1]{\textcolor[rgb]{0.37,0.37,0.37}{#1}}
\newcommand{\KeywordTok}[1]{\textcolor[rgb]{0.00,0.23,0.31}{\textbf{#1}}}
\newcommand{\NormalTok}[1]{\textcolor[rgb]{0.00,0.23,0.31}{#1}}
\newcommand{\OperatorTok}[1]{\textcolor[rgb]{0.37,0.37,0.37}{#1}}
\newcommand{\OtherTok}[1]{\textcolor[rgb]{0.00,0.23,0.31}{#1}}
\newcommand{\PreprocessorTok}[1]{\textcolor[rgb]{0.68,0.00,0.00}{#1}}
\newcommand{\RegionMarkerTok}[1]{\textcolor[rgb]{0.00,0.23,0.31}{#1}}
\newcommand{\SpecialCharTok}[1]{\textcolor[rgb]{0.37,0.37,0.37}{#1}}
\newcommand{\SpecialStringTok}[1]{\textcolor[rgb]{0.13,0.47,0.30}{#1}}
\newcommand{\StringTok}[1]{\textcolor[rgb]{0.13,0.47,0.30}{#1}}
\newcommand{\VariableTok}[1]{\textcolor[rgb]{0.07,0.07,0.07}{#1}}
\newcommand{\VerbatimStringTok}[1]{\textcolor[rgb]{0.13,0.47,0.30}{#1}}
\newcommand{\WarningTok}[1]{\textcolor[rgb]{0.37,0.37,0.37}{\textit{#1}}}

\usepackage{longtable,booktabs,array}
\usepackage{calc} % for calculating minipage widths
% Correct order of tables after \paragraph or \subparagraph
\usepackage{etoolbox}
\makeatletter
\patchcmd\longtable{\par}{\if@noskipsec\mbox{}\fi\par}{}{}
\makeatother
% Allow footnotes in longtable head/foot
\IfFileExists{footnotehyper.sty}{\usepackage{footnotehyper}}{\usepackage{footnote}}
\makesavenoteenv{longtable}
\usepackage{graphicx}
\makeatletter
\newsavebox\pandoc@box
\newcommand*\pandocbounded[1]{% scales image to fit in text height/width
  \sbox\pandoc@box{#1}%
  \Gscale@div\@tempa{\textheight}{\dimexpr\ht\pandoc@box+\dp\pandoc@box\relax}%
  \Gscale@div\@tempb{\linewidth}{\wd\pandoc@box}%
  \ifdim\@tempb\p@<\@tempa\p@\let\@tempa\@tempb\fi% select the smaller of both
  \ifdim\@tempa\p@<\p@\scalebox{\@tempa}{\usebox\pandoc@box}%
  \else\usebox{\pandoc@box}%
  \fi%
}
% Set default figure placement to htbp
\def\fps@figure{htbp}
\makeatother


% definitions for citeproc citations
\NewDocumentCommand\citeproctext{}{}
\NewDocumentCommand\citeproc{mm}{%
  \begingroup\def\citeproctext{#2}\cite{#1}\endgroup}
\makeatletter
 % allow citations to break across lines
 \let\@cite@ofmt\@firstofone
 % avoid brackets around text for \cite:
 \def\@biblabel#1{}
 \def\@cite#1#2{{#1\if@tempswa , #2\fi}}
\makeatother
\newlength{\cslhangindent}
\setlength{\cslhangindent}{1.5em}
\newlength{\csllabelwidth}
\setlength{\csllabelwidth}{3em}
\newenvironment{CSLReferences}[2] % #1 hanging-indent, #2 entry-spacing
 {\begin{list}{}{%
  \setlength{\itemindent}{0pt}
  \setlength{\leftmargin}{0pt}
  \setlength{\parsep}{0pt}
  % turn on hanging indent if param 1 is 1
  \ifodd #1
   \setlength{\leftmargin}{\cslhangindent}
   \setlength{\itemindent}{-1\cslhangindent}
  \fi
  % set entry spacing
  \setlength{\itemsep}{#2\baselineskip}}}
 {\end{list}}
\usepackage{calc}
\newcommand{\CSLBlock}[1]{\hfill\break\parbox[t]{\linewidth}{\strut\ignorespaces#1\strut}}
\newcommand{\CSLLeftMargin}[1]{\parbox[t]{\csllabelwidth}{\strut#1\strut}}
\newcommand{\CSLRightInline}[1]{\parbox[t]{\linewidth - \csllabelwidth}{\strut#1\strut}}
\newcommand{\CSLIndent}[1]{\hspace{\cslhangindent}#1}



\setlength{\emergencystretch}{3em} % prevent overfull lines

\providecommand{\tightlist}{%
  \setlength{\itemsep}{0pt}\setlength{\parskip}{0pt}}



 


\KOMAoption{captions}{tableheading}
\makeatletter
\@ifpackageloaded{bookmark}{}{\usepackage{bookmark}}
\makeatother
\makeatletter
\@ifpackageloaded{caption}{}{\usepackage{caption}}
\AtBeginDocument{%
\ifdefined\contentsname
  \renewcommand*\contentsname{Table of contents}
\else
  \newcommand\contentsname{Table of contents}
\fi
\ifdefined\listfigurename
  \renewcommand*\listfigurename{List of Figures}
\else
  \newcommand\listfigurename{List of Figures}
\fi
\ifdefined\listtablename
  \renewcommand*\listtablename{List of Tables}
\else
  \newcommand\listtablename{List of Tables}
\fi
\ifdefined\figurename
  \renewcommand*\figurename{Figure}
\else
  \newcommand\figurename{Figure}
\fi
\ifdefined\tablename
  \renewcommand*\tablename{Table}
\else
  \newcommand\tablename{Table}
\fi
}
\@ifpackageloaded{float}{}{\usepackage{float}}
\floatstyle{ruled}
\@ifundefined{c@chapter}{\newfloat{codelisting}{h}{lop}}{\newfloat{codelisting}{h}{lop}[chapter]}
\floatname{codelisting}{Listing}
\newcommand*\listoflistings{\listof{codelisting}{List of Listings}}
\makeatother
\makeatletter
\makeatother
\makeatletter
\@ifpackageloaded{caption}{}{\usepackage{caption}}
\@ifpackageloaded{subcaption}{}{\usepackage{subcaption}}
\makeatother
\usepackage{bookmark}
\IfFileExists{xurl.sty}{\usepackage{xurl}}{} % add URL line breaks if available
\urlstyle{same}
\hypersetup{
  pdftitle={my-ebook},
  pdfauthor={Parmeshvar},
  colorlinks=true,
  linkcolor={blue},
  filecolor={Maroon},
  citecolor={Blue},
  urlcolor={Blue},
  pdfcreator={LaTeX via pandoc}}


\title{my-ebook}
\author{Parmeshvar}
\date{2025-07-06}
\begin{document}
\maketitle

\renewcommand*\contentsname{Table of contents}
{
\hypersetup{linkcolor=}
\setcounter{tocdepth}{2}
\tableofcontents
}

\bookmarksetup{startatroot}

\chapter{Introduction}\label{introduction}

\bookmarksetup{startatroot}

\chapter{Introduction}\label{introduction-1}

\textbf{DR.Harsh Pradhan},
\href{http://www.ling.uni-potsdam.de/en/}{Phone: +91-9930034241 , Email:
harsh.231284@gmail.com},
\href{http://www.uni-potsdam.de/en/university-of-potsdam.html}{Institute
of Management Studies, Banaras Hindu University}, Address: 18-GF,
Jaipuria Enclave, Kaushambhi, Ghaziabad, India, 201010\\
\textbf{Interest}:
\href{http://www.uni-potsdam.de/humfak/hum-forschungsschwerpunkte/forschungscluster-sprache.html}{Goal
Orientation Job Performance Consumer Behavior Behavioral Finance
Bibiliometric Analysis Options as Derivatives Statistics Indian
Knowledge System},

\href{https://orcid.org/0000-0002-3332-3610}{Orcid ID}\\
\href{https://scholar.google.com/citations?user=8l5MEd0AAAAJ&hl=en&oi=sra}{Google
Scholar}\\
\href{https://github.com/}{GitHub}\\
\href{https://www.webofscience.com/wos/author/record/AIB}{Researcher
ID}\\
\href{http://https//sites.google.com/view/harshpradhan/home\%20\%20,\%20\%20\%20\%20\%20shorturl.at/yzQSX}{Personal
Website}\\
\href{http://www.youtube.com/@dr.harshpradhan742}{Youtube ID}

\textbf{Doing a PhD with me:}
\href{http://www.ling.uni-potsdam.de/~vasishth/doingaphdwithme.html}{README.1st}

\href{https://bhu.ac.in/Site/FacultyProfile/1_5?FA000562}{Academic
Profile}

\begin{center}\rule{0.5\linewidth}{0.5pt}\end{center}

\bookmarksetup{startatroot}

\chapter{Bayesian data analysis for cognitive
science}\label{bayesian-data-analysis-for-cognitive-science}

\section{Introduction: What this course is
about}\label{introduction-what-this-course-is-about}

This course provides an introduction to Bayesian data analysis using the
probabilistic programming language \textbf{Stan}.\\
We will use a front end software package called \textbf{brms}.

This course is for:

\begin{itemize}
\tightlist
\item
  Linguistics (MM5, MM6)
\item
  Cognitive Systems
\item
  Cognitive Science
\end{itemize}

Please see the
\href{http://stud.astaup.de/~linguistik/doku.php?id=puls-faq}{PULS FAQs}
to find out how the sign-up system works (in German).

We will be using the software \href{http://cran.r-project.org/}{R} and
\href{https://www.rstudio.com/}{RStudio}, so make sure you install these
on your computer.

Topics to be covered:

\begin{enumerate}
\def\labelenumi{\arabic{enumi}.}
\tightlist
\item
  Basic probability theory, random variable theory (including jointly
  distributed RVs), probability distributions (including bivariate
  distributions)
\item
  Using Bayes' rule for statistical inference
\item
  An introduction to (generalized) linear models
\item
  An introduction to hierarchical models
\item
  Measurement error models
\item
  Mixture models
\item
  Model selection and hypothesis testing (Bayes factor and k-fold
  cross-validation)
\end{enumerate}

\section{Teaching}\label{teaching}

Science and statistics is/are one unitary thing; you cannot do one
without the other. Towards this end, I teach some (in my opinion)
critically important classes that provide a solid statistical foundation
for doing research in cognitive science. Free online course, four weeks
(MOOC), enrollments open: Introduction to Bayesian Data Analysis. Short
(four-hour) tutorial on Bayesian statistics, taught at EMLAR 2022: here
Introduction to (frequentist) statistics Introduction to Bayesian data
analysis for cognitive science BDA cover

\section{Lecture notes}\label{lecture-notes}

Download from
\href{https://drive.google.com/drive/folders/1H-jQNOqS3PLgdXADRYF7ag-WqN_tK4qc?usp=share_link}{here}.

\section{Moodle website}\label{moodle-website}

All communications with students in Potsdam will be done through
\href{https://moodle2.uni-potsdam.de/course/view.php?id=17526}{this
website}.

\bookmarksetup{startatroot}

\chapter{Schedule}\label{schedule}

\begin{longtable}[]{@{}
  >{\raggedright\arraybackslash}p{(\linewidth - 10\tabcolsep) * \real{0.0610}}
  >{\raggedright\arraybackslash}p{(\linewidth - 10\tabcolsep) * \real{0.0793}}
  >{\raggedright\arraybackslash}p{(\linewidth - 10\tabcolsep) * \real{0.1951}}
  >{\raggedright\arraybackslash}p{(\linewidth - 10\tabcolsep) * \real{0.2073}}
  >{\raggedright\arraybackslash}p{(\linewidth - 10\tabcolsep) * \real{0.2195}}
  >{\raggedright\arraybackslash}p{(\linewidth - 10\tabcolsep) * \real{0.2378}}@{}}
\toprule\noalign{}
\begin{minipage}[b]{\linewidth}\raggedright
\textbf{Week}
\end{minipage} & \begin{minipage}[b]{\linewidth}\raggedright
\textbf{Lecture}
\end{minipage} & \begin{minipage}[b]{\linewidth}\raggedright
\textbf{Main Topic}
\end{minipage} & \begin{minipage}[b]{\linewidth}\raggedright
\textbf{Sub Topic}
\end{minipage} & \begin{minipage}[b]{\linewidth}\raggedright
\textbf{Video}
\end{minipage} & \begin{minipage}[b]{\linewidth}\raggedright
\textbf{PDF Resource}
\end{minipage} \\
\midrule\noalign{}
\endhead
\bottomrule\noalign{}
\endlastfoot
Jan 30 + Feb 4 & - & Model Selection \& Hypothesis Testing & - & - &
\hyperref[]{HW 13} \\
Week 2 & 1 & Descriptive Statistics & Central Tendency &
\hyperref[]{Link} &
\href{https://drive.google.com/drive/u/0/folders/1_-dD19gdBcDVGIpBiMt1ChTZ-6z3lmw0}{Week
2.pdf} \\
& 2 & Descriptive Statistics & Measure of Variability &
\hyperref[]{Link} &
\href{https://drive.google.com/drive/u/0/folders/1_-dD19gdBcDVGIpBiMt1ChTZ-6z3lmw0}{Week
2.pdf} \\
& 3 & Descriptive Statistics & Describing Data & \hyperref[]{Link} &
\href{https://drive.google.com/drive/u/0/folders/1_-dD19gdBcDVGIpBiMt1ChTZ-6z3lmw0}{Week
2.pdf} \\
& 4 & Probability & - & \hyperref[]{Link} &
\href{https://drive.google.com/drive/u/0/folders/1_-dD19gdBcDVGIpBiMt1ChTZ-6z3lmw0}{Week
2.pdf} \\
& 5 & Distribution & - & \hyperref[]{Link} &
\href{https://drive.google.com/drive/u/0/folders/1_-dD19gdBcDVGIpBiMt1ChTZ-6z3lmw0}{Week
2.pdf} \\
Week 3 & 1 & Probability & Z Table (Normal Distribution) &
\hyperref[]{Link} &
\href{https://drive.google.com/drive/u/0/folders/1_-dD19gdBcDVGIpBiMt1ChTZ-6z3lmw0}{Week
3.pdf} \\
& 2 & Divergence & Measuring Divergence & \hyperref[]{Link} &
\href{https://drive.google.com/drive/u/0/folders/1_-dD19gdBcDVGIpBiMt1ChTZ-6z3lmw0}{Week
3.pdf} \\
& 3 & Inferential Statistics & Sample and Population & \hyperref[]{Link}
&
\href{https://drive.google.com/drive/u/0/folders/1_-dD19gdBcDVGIpBiMt1ChTZ-6z3lmw0}{Week
3.pdf} \\
& 4 & Model Fit & - & \hyperref[]{Link} &
\href{https://drive.google.com/drive/u/0/folders/1_-dD19gdBcDVGIpBiMt1ChTZ-6z3lmw0}{Week
3.pdf} \\
& 5 & Hypothesis Testing & Hypothesis and Error & \hyperref[]{Link} &
\href{https://drive.google.com/drive/u/0/folders/1_-dD19gdBcDVGIpBiMt1ChTZ-6z3lmw0}{Week
3.pdf} \\
Week 4 & 1 & Statistical Terms & Terms of Statistics & \hyperref[]{Link}
&
\href{https://drive.google.com/drive/u/0/folders/1_-dD19gdBcDVGIpBiMt1ChTZ-6z3lmw0}{Week
4.pdf} \\
& 2 & Hypothesis Testing & T-Test & \hyperref[]{Link} &
\href{https://drive.google.com/drive/u/0/folders/1_-dD19gdBcDVGIpBiMt1ChTZ-6z3lmw0}{Week
4.pdf} \\
& 3 & Hypothesis Testing & T-Test in Detail & \hyperref[]{Link} &
\href{https://drive.google.com/drive/u/0/folders/1_-dD19gdBcDVGIpBiMt1ChTZ-6z3lmw0}{Week
4.pdf} \\
& 4 & ANOVA & ANOVA & \hyperref[]{Link} &
\href{https://drive.google.com/drive/u/0/folders/1_-dD19gdBcDVGIpBiMt1ChTZ-6z3lmw0}{Week
4.pdf} \\
Week 5 & 1 & ANOVA & Example of ANOVA & \hyperref[]{Link} &
\href{https://drive.google.com/drive/u/0/folders/1_-dD19gdBcDVGIpBiMt1ChTZ-6z3lmw0}{Week
5.pdf} \\
& 2 & ANOVA & Types of ANOVA & \hyperref[]{Link} &
\href{https://drive.google.com/drive/u/0/folders/1_-dD19gdBcDVGIpBiMt1ChTZ-6z3lmw0}{Week
5.pdf} \\
& 3 & Correlation & Introduction to Correlation & \hyperref[]{Link} &
\href{https://drive.google.com/drive/u/0/folders/1_-dD19gdBcDVGIpBiMt1ChTZ-6z3lmw0}{Week
5.pdf} \\
& 4 & Regression & Regression & \hyperref[]{Link} &
\href{https://drive.google.com/drive/u/0/folders/1_-dD19gdBcDVGIpBiMt1ChTZ-6z3lmw0}{Week
5.pdf} \\
& 5 & Regression & Regression & \hyperref[]{Link} &
\href{https://drive.google.com/drive/u/0/folders/1_-dD19gdBcDVGIpBiMt1ChTZ-6z3lmw0}{Week
5.pdf} \\
Week 6 & 1 & Regression & R Script for Regression & \hyperref[]{Link} &
\href{https://drive.google.com/drive/u/0/folders/1_-dD19gdBcDVGIpBiMt1ChTZ-6z3lmw0}{Week
6.pdf} \\
& 2 & Chi-Square & Chi Square & \hyperref[]{Link} &
\href{https://drive.google.com/drive/u/0/folders/1_-dD19gdBcDVGIpBiMt1ChTZ-6z3lmw0}{Week
6.pdf} \\
& 3 & Chi-Square & Chi Square Test & \hyperref[]{Link} &
\href{https://drive.google.com/drive/u/0/folders/1_-dD19gdBcDVGIpBiMt1ChTZ-6z3lmw0}{Week
6.pdf} \\
& 4 & Logistic Regression & Logistic Function & \hyperref[]{Link} &
\href{https://drive.google.com/drive/u/0/folders/1_-dD19gdBcDVGIpBiMt1ChTZ-6z3lmw0}{Week
6.pdf} \\
& 5 & Distribution & - & \hyperref[]{Link} &
\href{https://drive.google.com/drive/u/0/folders/1_-dD19gdBcDVGIpBiMt1ChTZ-6z3lmw0}{Week
6.pdf} \\
Week 7 & 1 & Time Series & Intro to Time Series & \hyperref[]{Link} &
\href{https://drive.google.com/drive/u/0/folders/1_-dD19gdBcDVGIpBiMt1ChTZ-6z3lmw0}{Week
7.pdf} \\
& 2 & Probability & Conditional Probability & \hyperref[]{Link} &
\href{https://drive.google.com/drive/u/0/folders/1_-dD19gdBcDVGIpBiMt1ChTZ-6z3lmw0}{Week
7.pdf} \\
& 3 & Additional Concepts & - & \hyperref[]{Link} &
\href{https://drive.google.com/drive/u/0/folders/1_-dD19gdBcDVGIpBiMt1ChTZ-6z3lmw0}{Week
7.pdf} \\
& 4 & Distribution & - & \hyperref[]{Link} &
\href{https://drive.google.com/drive/u/0/folders/1_-dD19gdBcDVGIpBiMt1ChTZ-6z3lmw0}{Week
7.pdf} \\
& 5 & Poisson Distribution & - & \hyperref[]{Link} &
\href{https://drive.google.com/drive/u/0/folders/1_-dD19gdBcDVGIpBiMt1ChTZ-6z3lmw0}{Week
7.pdf} \\
Week 8 & 1 & Libraries \& Documentation & Effect Size and Packages &
\hyperref[]{Link} &
\href{https://drive.google.com/drive/u/0/folders/1_-dD19gdBcDVGIpBiMt1ChTZ-6z3lmw0}{Week
8.pdf} \\
& 2 & Software Comparison & RStudio vs RKward & \hyperref[]{Link} &
\href{https://drive.google.com/drive/u/0/folders/1_-dD19gdBcDVGIpBiMt1ChTZ-6z3lmw0}{Week
8.pdf} \\
& 3 & Visualization & Flexplot & \hyperref[]{Link} &
\href{https://drive.google.com/drive/u/0/folders/1_-dD19gdBcDVGIpBiMt1ChTZ-6z3lmw0}{Week
8.pdf} \\
& 4 & Programming in R & Functions & \hyperref[]{Link} &
\href{https://drive.google.com/drive/u/0/folders/1_-dD19gdBcDVGIpBiMt1ChTZ-6z3lmw0}{Week
8.pdf} \\
& 5 & R Tools & R Shiny and R Markdown & \hyperref[]{Link} &
\href{https://drive.google.com/drive/u/0/folders/1_-dD19gdBcDVGIpBiMt1ChTZ-6z3lmw0}{Week
8.pdf} \\
\end{longtable}

\begin{center}\rule{0.5\linewidth}{0.5pt}\end{center}

\bookmarksetup{startatroot}

\chapter{Introduction to Statistics}\label{introduction-to-statistics}

\bookmarksetup{startatroot}

\chapter{Chapter 1: Welcome and Course
Overview}\label{chapter-1-welcome-and-course-overview}

This course offers an introduction to statistics through the RKWard
graphical interface of R. Aimed at learners from diverse backgrounds,
the course emphasizes practical application over theory. You don't need
a strong background in math or computing---just an eagerness to learn.

\textbf{Pre-Requisites:}

\begin{itemize}
\tightlist
\item
  Curiosity\\
\item
  Basic awareness of numbers\\
\item
  No fear of statistics or software
\end{itemize}

\begin{quote}
``Aapko darne ki zarurat nahi hai\ldots{} simple understanding aapko
statistics ki data ki aage milegi.''
\end{quote}

\bookmarksetup{startatroot}

\chapter{Chapter 2: Agenda and
Orientation}\label{chapter-2-agenda-and-orientation}

\textbf{Key Themes:}

\begin{itemize}
\tightlist
\item
  Difference between Mathematics and Statistics\\
\item
  Nature, Meaning, and Role of Statistics\\
\item
  Uses, Limitations, and Common Fallacies
\end{itemize}

\begin{longtable}[]{@{}
  >{\raggedright\arraybackslash}p{(\linewidth - 4\tabcolsep) * \real{0.1375}}
  >{\raggedright\arraybackslash}p{(\linewidth - 4\tabcolsep) * \real{0.3250}}
  >{\raggedright\arraybackslash}p{(\linewidth - 4\tabcolsep) * \real{0.5375}}@{}}
\toprule\noalign{}
\begin{minipage}[b]{\linewidth}\raggedright
Aspect
\end{minipage} & \begin{minipage}[b]{\linewidth}\raggedright
Mathematics
\end{minipage} & \begin{minipage}[b]{\linewidth}\raggedright
Statistics
\end{minipage} \\
\midrule\noalign{}
\endhead
\bottomrule\noalign{}
\endlastfoot
Nature & Abstract, theoretical & Applied, data-centric \\
Focus & Concepts, theorems, proofs & Tools, interpretation,
decision-making \\
Tools & Logical reasoning, algebra & Hypothesis testing, regression,
probability \\
Application & General structures & Real-world problems \\
\end{longtable}

\bookmarksetup{startatroot}

\chapter{Chapter 3: Meaning and Nature of
Statistics}\label{chapter-3-meaning-and-nature-of-statistics}

\textbf{Definition:}\\
Statistics is the science of collecting, analyzing, interpreting, and
presenting data for decision-making.

\textbf{Core Concepts:}

\begin{itemize}
\tightlist
\item
  Population \& Sample\\
\item
  Parameter \& Statistic\\
\item
  Data classification and tabulation
\end{itemize}

\textbf{Purpose:}

\begin{itemize}
\tightlist
\item
  Describe and explain phenomena\\
\item
  Interpret and predict outcomes\\
\item
  Facilitate scientific and social inquiry
\end{itemize}

\bookmarksetup{startatroot}

\chapter{Chapter 4: Applications and
Uses}\label{chapter-4-applications-and-uses}

\textbf{Main Uses:}

\begin{itemize}
\tightlist
\item
  Summarizing observed data\\
\item
  Drawing representative samples\\
\item
  Analyzing relationships and trends\\
\item
  Supporting decision-making in fields like marketing, psychology,
  education, and public health
\end{itemize}

\textbf{Important Concepts:}

\begin{itemize}
\tightlist
\item
  Data summarization\\
\item
  Prediction based on patterns\\
\item
  Comparison across groups\\
\item
  Scientific objectivity
\end{itemize}

\bookmarksetup{startatroot}

\chapter{Chapter 5: Limitations and
Misuse}\label{chapter-5-limitations-and-misuse}

\textbf{Limitations:}

\begin{itemize}
\tightlist
\item
  Cannot analyze qualitative phenomena\\
\item
  Not designed for individuals\\
\item
  Results aren't exact\\
\item
  Misinterpretation leads to incorrect conclusions
\end{itemize}

\textbf{Misuse Includes:}

\begin{itemize}
\tightlist
\item
  Small or biased samples\\
\item
  Misleading graphs\\
\item
  Invalid comparisons
\end{itemize}

\begin{quote}
``Statistics is not a substitute for common sense or understanding the
context.''
\end{quote}

\textbf{Fallacies Stem From:}

\begin{itemize}
\tightlist
\item
  Poor data collection\\
\item
  Mislabeling variables\\
\item
  Improper classification or selection
\end{itemize}

\bookmarksetup{startatroot}

\chapter{Chapter 6: Paper-Based vs.~Software-Based
Statistics}\label{chapter-6-paper-based-vs.-software-based-statistics}

Traditional exams test pen-paper knowledge, but software-based tools
like RKWard make analysis:

\begin{itemize}
\tightlist
\item
  Faster\\
\item
  Collaborative\\
\item
  Easier to store and access\\
\item
  Essential for modern data-centric fields like AI and machine learning
\end{itemize}

Understanding both paper and digital approaches ensures comprehensive
learning.

\bookmarksetup{startatroot}

\chapter{Chapter 7: Introduction to Variables and
Spreadsheets}\label{chapter-7-introduction-to-variables-and-spreadsheets}

\textbf{Variables:}

\begin{itemize}
\tightlist
\item
  Store information (e.g., \texttt{x\ =\ 5})\\
\item
  Have unique names\\
\item
  Can be manipulated with commands (e.g., \texttt{x\ =\ x\ +\ 2})
\end{itemize}

\textbf{Spreadsheets:}

\begin{itemize}
\tightlist
\item
  Represent tabular data (rows = observations, columns = variables)\\
\item
  Familiar formats: Excel, Google Sheets\\
\item
  Essential in statistical packages
\end{itemize}

\bookmarksetup{startatroot}

\chapter{Chapter 8: R and GUI
Interfaces}\label{chapter-8-r-and-gui-interfaces}

\textbf{Why R?:}

\begin{itemize}
\tightlist
\item
  Free and open-source\\
\item
  Strong community support\\
\item
  High flexibility\\
\item
  Powerful graphics and data manipulation capabilities
\end{itemize}

\textbf{GUI Tools in R:}

\begin{itemize}
\tightlist
\item
  RKWard \emph{(used in this course)}\\
\item
  R Commander\\
\item
  Rattle\\
\item
  R AnalyticFlow
\end{itemize}

\textbf{Basic Terms:}

\begin{itemize}
\tightlist
\item
  \textbf{Console}: Type commands \& view outputs\\
\item
  \textbf{Working Directory}: File storage location\\
\item
  \textbf{Package}: Predefined or custom functions\\
\item
  \textbf{Script}: Collection of reusable commands\\
\item
  \textbf{Workspace}: All current variables/functions
\end{itemize}

\bookmarksetup{startatroot}

\chapter{Chapter 9: Importing Data and Understanding Data
Types}\label{chapter-9-importing-data-and-understanding-data-types}

\textbf{Using RKWard:}

\begin{itemize}
\tightlist
\item
  Import CSV files using GUI\\
\item
  Data appears in alphabetical order in workspace\\
\item
  Each header = variable name
\end{itemize}

\textbf{Data Structures:}

\begin{itemize}
\tightlist
\item
  Data Frames (most commonly used)\\
\item
  Matrices\\
\item
  Vectors\\
\item
  Lists
\end{itemize}

\textbf{Command Line vs GUI:}

\begin{itemize}
\tightlist
\item
  Both achieve the same results\\
\item
  GUI is user-friendly, command line is customizable
\end{itemize}

\begin{Shaded}
\begin{Highlighting}[]
\FunctionTok{mean}\NormalTok{(my\_csv.data}\SpecialCharTok{$}\NormalTok{JP\_01)  }\CommentTok{\# Calculates the mean of variable JP\_01}
\end{Highlighting}
\end{Shaded}

\bookmarksetup{startatroot}

\chapter{Chapter 10: Statistical Data
Types}\label{chapter-10-statistical-data-types}

\begin{longtable}[]{@{}
  >{\raggedright\arraybackslash}p{(\linewidth - 4\tabcolsep) * \real{0.2105}}
  >{\raggedright\arraybackslash}p{(\linewidth - 4\tabcolsep) * \real{0.6316}}
  >{\raggedright\arraybackslash}p{(\linewidth - 4\tabcolsep) * \real{0.1579}}@{}}
\toprule\noalign{}
\begin{minipage}[b]{\linewidth}\raggedright
Statistical Type
\end{minipage} & \begin{minipage}[b]{\linewidth}\raggedright
Description
\end{minipage} & \begin{minipage}[b]{\linewidth}\raggedright
R Equivalent
\end{minipage} \\
\midrule\noalign{}
\endhead
\bottomrule\noalign{}
\endlastfoot
Nominal & Names, labels (e.g., Male/Female) & String \\
Ordinal & Order/rank (e.g., 1st, 2nd) & Factor \\
Interval & Ordered + meaningful intervals (e.g., tax slabs) & Numeric \\
Ratio & Includes absolute zero (e.g., weight) & Numeric \\
\end{longtable}

\textbf{Others in R:}

\begin{itemize}
\tightlist
\item
  Logical (TRUE/FALSE)\\
\item
  Integer, Complex
\end{itemize}

\begin{quote}
Remember: Not all numbers mean quantity. Shirt numbers (like \#18) are
nominal, not mathematical.
\end{quote}

\bookmarksetup{startatroot}

\chapter{Chapter 11: Data Preparation in
RKWard}\label{chapter-11-data-preparation-in-rkward}

\begin{itemize}
\tightlist
\item
  Data must be properly \textbf{typed} (e.g., ``1'' as number vs ``1''
  as label)\\
\item
  Check alignment: Left = character, Right = number\\
\item
  \textbf{Labels} help collaborators understand variables\\
\item
  Example: \texttt{Gender\ =\ 1} (Male), \texttt{0} (Female)\\
\item
  Must distinguish between numeric calculations and categorical
  identifiers
\end{itemize}

\textbf{Best Practices:}

\begin{itemize}
\tightlist
\item
  Define each variable with meaning\\
\item
  Validate data types\\
\item
  Store and share workspace for reproducibility
\end{itemize}

\bookmarksetup{startatroot}

\chapter{Chapter 12: Visualizing Data with Plots in
RKWard}\label{chapter-12-visualizing-data-with-plots-in-rkward}

Data visualization is essential to reveal patterns, trends, and
distributions. RKWard offers multiple graphical tools:

\section{1. Histogram}\label{histogram}

\begin{itemize}
\tightlist
\item
  Depicts the distribution of a single variable\\
\item
  Can include frequency, relative frequency, and cumulative frequency\\
\item
  Best for understanding where most data points lie
\end{itemize}

\section{2. Pie Chart}\label{pie-chart}

\begin{itemize}
\tightlist
\item
  Represents categorical data as slices of a circle\\
\item
  Best when visualizing proportions
\end{itemize}

\section{3. Scatter Plot}\label{scatter-plot}

\begin{itemize}
\tightlist
\item
  Plots two variables to examine relationships\\
\item
  X-axis: Independent variable\\
\item
  Y-axis: Dependent variable\\
\item
  Useful in exploring associations or potential causality
\end{itemize}

\section{4. Box Plot}\label{box-plot}

\begin{itemize}
\tightlist
\item
  Shows data distribution via quartiles\\
\item
  Median, interquartile range (IQR), and outliers are clearly
  indicated\\
\item
  Useful for comparing multiple variables
\end{itemize}

\section{5. Density Plot}\label{density-plot}

\begin{itemize}
\tightlist
\item
  Smoothed version of a histogram\\
\item
  Better suited for continuous data with decimal variation
\end{itemize}

\textbf{Key Tips:}

\begin{itemize}
\tightlist
\item
  \texttt{JP\_01} was frequently used as an example variable\\
\item
  RKWard allows saving and exporting plots easily\\
\item
  GUI menus guide the user through plot creation
\end{itemize}

\begin{quote}
Always choose the plot type that best matches your data and goal:
frequency, relationship, or comparison.
\end{quote}

\bookmarksetup{startatroot}

\chapter{Chapter 13: Summary}\label{chapter-13-summary}

This eBook provided a foundation for understanding and applying
statistics using the RKWard GUI tool in R. It covered essential concepts
from what statistics is, to importing and handling data, understanding
types of variables and their measurement levels, and visualizing data
using a variety of plots.

Learners were introduced to:

\begin{itemize}
\tightlist
\item
  Basic statistical principles\\
\item
  Software versus paper-based understanding\\
\item
  Variable types and spreadsheet usage\\
\item
  Command line and GUI-based tools\\
\item
  Data visualization through histogram, pie, scatter, box, and density
  plots
\end{itemize}

The course emphasized \textbf{conceptual clarity}, \textbf{practical
tools}, and the \textbf{power of visualization}. It prepares learners to
interpret, analyze, and present data meaningfully in academic or
real-world contexts.

\bookmarksetup{startatroot}

\chapter{References}\label{references}

\begin{enumerate}
\def\labelenumi{\arabic{enumi}.}
\tightlist
\item
  Mohanty, B., \& Misra, S. (2020). \emph{Statistics for Behavioral and
  Social Sciences}. PHI Learning.\\
\item
  Pandya, D., et al.~(2019). \emph{Statistical Analysis in Simple Steps
  Using R}. Wiley.\\
\item
  Field, A., Miles, J., \& Field, Z. (2012). \emph{Discovering
  Statistics Using R}. SAGE Publications.\\
\item
  Harris, J. (2021). \emph{Statistics with R: Solving Problems Using
  Real-World Data}. Pearson.\\
\item
  RKWard Project: \url{https://rkward.kde.org}
\end{enumerate}

\bookmarksetup{startatroot}

\chapter{Next Steps}\label{next-steps}

Upcoming lectures will cover:

\begin{itemize}
\tightlist
\item
  Graph creation\\
\item
  Data visualization tools\\
\item
  Advanced statistical operations in GUI
\end{itemize}

\bookmarksetup{startatroot}

\chapter{basic-statistics\_1}\label{basic-statistics_1}

\bookmarksetup{startatroot}

\chapter{Introduction}\label{introduction-2}

Welcome to the ``Basic Statistics Using GUI-R (RK Ward)'' course, led by
Dr.~Harsh Pradhan at the Institute of Management Studies, Banaras Hindu
University. This course takes an integrated approach to statistical
analysis, bridging theory with practical skills through the R
programming language and its GUI, RKWard.

\section{Objectives of the Course}\label{objectives-of-the-course}

\begin{itemize}
\tightlist
\item
  Understand fundamental concepts related to statistics.
\item
  Gain proficiency in using R and RKWard for statistical analysis.
\item
  Learn to visualize data effectively.
\item
  Apply statistical methodologies to real-world datasets.
\end{itemize}

\bookmarksetup{startatroot}

\chapter{Overview of R and RKWard}\label{overview-of-r-and-rkward}

\section{R Programming Language}\label{r-programming-language}

R is a versatile, open-source language specifically designed for
statistical analysis and data visualization. It provides an extensive
suite of statistical procedures, making it a cornerstone for
statisticians and data scientists.

\textbf{Key Features of R:}

\begin{itemize}
\tightlist
\item
  \textbf{Extensive Libraries:} R hosts thousands of packages that
  support numerous statistical models such as linear regression, time
  series, and more.
\item
  \textbf{Customizable Graphics:} The base graphics capabilities, along
  with packages like \texttt{ggplot2}, allow users to create a variety
  of complex visualizations with relative ease.
\item
  \textbf{Data Manipulation Tools:} Packages like \texttt{dplyr} and
  \texttt{tidyr} provide robust tools for data cleaning and
  transformation.
\end{itemize}

\section{Understanding RKWard}\label{understanding-rkward}

RKWard serves as a user-friendly interface that simplifies interactions
with R, allowing users---especially those less familiar with
programming---to utilize its powerful capabilities without a steep
learning curve.

\textbf{Features of RKWard Include:}

\begin{itemize}
\tightlist
\item
  \textbf{Graphical User Interface:} Navigation through menus rather
  than command lines enhances accessibility.
\item
  \textbf{Built-in Documentation:} Context-sensitive help facilitates
  learning and troubleshooting.
\item
  \textbf{Integration with R:} Commands executed via the GUI can be
  viewed and modified, providing a dual-learning experience.
\end{itemize}

\bookmarksetup{startatroot}

\chapter{Understanding Variables}\label{understanding-variables}

\section{Types of Variables}\label{types-of-variables}

Variables are the building blocks of statistical analysis, representing
the characteristics or properties of the data.

\subsection{Qualitative Variables (Categorical
Variables)}\label{qualitative-variables-categorical-variables}

\begin{itemize}
\tightlist
\item
  \textbf{Nominal Variables:} These variables categorize data without an
  inherent order. For example, types of fruits (apple, orange) are
  nominal.
\item
  \textbf{Ordinal Variables:} These represent ordered categories. For
  instance, a customer satisfaction survey may be rated as poor, fair,
  good, or excellent.
\end{itemize}

\subsection{Quantitative Variables}\label{quantitative-variables}

\begin{itemize}
\tightlist
\item
  \textbf{Discrete Variables:} These variables take on countable values,
  such as the number of students in a class.
\item
  \textbf{Continuous Variables:} These can take any value within a given
  range, such as height and weight.
\end{itemize}

\section{Importance of Defining
Variables}\label{importance-of-defining-variables}

Properly understanding and defining variables is crucial for:

\begin{itemize}
\tightlist
\item
  Selecting appropriate statistical tests.
\item
  Ensuring accurate data interpretation.
\item
  Structuring datasets to facilitate analysis.
\end{itemize}

\bookmarksetup{startatroot}

\chapter{Data Types and Spreadsheet
Concepts}\label{data-types-and-spreadsheet-concepts}

\section{Statistical Data Types}\label{statistical-data-types}

Data types are foundational for statistical analysis as they define what
kind of arithmetic operations can be performed on the data.

\begin{longtable}[]{@{}
  >{\raggedright\arraybackslash}p{(\linewidth - 4\tabcolsep) * \real{0.1250}}
  >{\raggedright\arraybackslash}p{(\linewidth - 4\tabcolsep) * \real{0.5208}}
  >{\raggedright\arraybackslash}p{(\linewidth - 4\tabcolsep) * \real{0.3542}}@{}}
\toprule\noalign{}
\begin{minipage}[b]{\linewidth}\raggedright
Data Type
\end{minipage} & \begin{minipage}[b]{\linewidth}\raggedright
Description
\end{minipage} & \begin{minipage}[b]{\linewidth}\raggedright
Example
\end{minipage} \\
\midrule\noalign{}
\endhead
\bottomrule\noalign{}
\endlastfoot
Nominal & Categorical data without order & Blood types (A, B, AB, O) \\
Ordinal & Categorical data with a defined order & Customer satisfaction
(poor, fair, good) \\
Interval & Numerical data with meaningful differences & Temperature in
Celsius \\
Ratio & Numerical data with an absolute zero & Weight, height \\
\end{longtable}

\section{Spreadsheet Basics}\label{spreadsheet-basics}

Spreadsheets provide a structured format for data entry, where rows
represent instances (e.g., individuals, items) and columns represent
variables (e.g., age, gender).

\textbf{Key Functions of Spreadsheets:}

\begin{itemize}
\tightlist
\item
  \textbf{Data Organization:} Data is easily sorted and filtered.
\item
  \textbf{Formulas and Functions:} Built-in functions allow for quick
  calculation and data manipulation.
\item
  \textbf{Visualization Integration:} Charts and tables can visually
  represent data.
\end{itemize}

\bookmarksetup{startatroot}

\chapter{Importing Data in RKWard}\label{importing-data-in-rkward}

\section{Data Preparation}\label{data-preparation}

Before importing data into RKWard, ensure that your dataset meets
standards such as:

\begin{itemize}
\tightlist
\item
  Properly labeled columns.
\item
  Consistent data types.
\item
  Absence of unnecessary formatting or symbols.
\end{itemize}

\section{Step-by-Step Import Process}\label{step-by-step-import-process}

To import data into RKWard, follow these steps:

\begin{enumerate}
\def\labelenumi{\arabic{enumi}.}
\tightlist
\item
  \textbf{Open RKWard} and access the main interface.
\item
  \textbf{Navigate to the ``Data'' Tab:}

  \begin{itemize}
  \tightlist
  \item
    Select ``Import Data'' option.
  \end{itemize}
\item
  \textbf{Choose the File Type:}

  \begin{itemize}
  \tightlist
  \item
    Select the file format (e.g., CSV, Excel) of your prepared dataset.
  \end{itemize}
\item
  \textbf{Browse for Your File:}

  \begin{itemize}
  \tightlist
  \item
    Utilize the ``Browse'' button to locate the file on your system.
  \end{itemize}
\item
  \textbf{Mapping Variables:}

  \begin{itemize}
  \tightlist
  \item
    During the import process, specify the data types for each column.
  \item
    Ensure the first row is used as header labels.
  \end{itemize}
\item
  \textbf{Review and Confirm Import:}

  \begin{itemize}
  \tightlist
  \item
    Once the data is imported, review it in the RKWard workspace,
    ensuring everything is correctly represented.
  \end{itemize}
\end{enumerate}

\bookmarksetup{startatroot}

\chapter{Basic Statistical Practices}\label{basic-statistical-practices}

\section{Descriptive Statistics}\label{descriptive-statistics}

Descriptive statistics help summarize and organize data in a meaningful
way.

\subsection{Central Tendency Measures}\label{central-tendency-measures}

\begin{itemize}
\tightlist
\item
  \textbf{Mean:} Average of the dataset.
\item
  \textbf{Median:} Middle value when data is ordered.
\item
  \textbf{Mode:} Most frequent value in the dataset.
\end{itemize}

\begin{longtable}[]{@{}
  >{\raggedright\arraybackslash}p{(\linewidth - 4\tabcolsep) * \real{0.0989}}
  >{\raggedright\arraybackslash}p{(\linewidth - 4\tabcolsep) * \real{0.4835}}
  >{\raggedright\arraybackslash}p{(\linewidth - 4\tabcolsep) * \real{0.4176}}@{}}
\toprule\noalign{}
\begin{minipage}[b]{\linewidth}\raggedright
Measure
\end{minipage} & \begin{minipage}[b]{\linewidth}\raggedright
Formula
\end{minipage} & \begin{minipage}[b]{\linewidth}\raggedright
Description
\end{minipage} \\
\midrule\noalign{}
\endhead
\bottomrule\noalign{}
\endlastfoot
Mean & \(\bar{x} = \frac{\sum_{i=1}^{n} x_i}{n}\) & Average value \\
Median & (Sorted data, middle item) & Middle value in ordered dataset \\
Mode & Value that appears most frequently & Most common value \\
\end{longtable}

\subsection{Dispersion Measures}\label{dispersion-measures}

\begin{itemize}
\tightlist
\item
  \textbf{Range:} Difference between the maximum and minimum values.
\item
  \textbf{Variance:} Measurement of the spread of data points.
\item
  \textbf{Standard Deviation:} Square root of variance, providing a
  measure of the average distance from the mean.
\end{itemize}

\begin{longtable}[]{@{}
  >{\raggedright\arraybackslash}p{(\linewidth - 4\tabcolsep) * \real{0.2019}}
  >{\raggedright\arraybackslash}p{(\linewidth - 4\tabcolsep) * \real{0.4712}}
  >{\raggedright\arraybackslash}p{(\linewidth - 4\tabcolsep) * \real{0.3269}}@{}}
\toprule\noalign{}
\begin{minipage}[b]{\linewidth}\raggedright
Measure
\end{minipage} & \begin{minipage}[b]{\linewidth}\raggedright
Formula
\end{minipage} & \begin{minipage}[b]{\linewidth}\raggedright
Description
\end{minipage} \\
\midrule\noalign{}
\endhead
\bottomrule\noalign{}
\endlastfoot
Range & \(Range = Max - Min\) & Spread of dataset \\
Variance & \(Var(X) = \frac{\sum_{i=1}^{n} (x_i - \bar{x})^2}{n - 1}\) &
Spread of data relative to mean \\
Standard Deviation & \(SD(X) = \sqrt{Var(X)}\) & Average distance from
mean \\
\end{longtable}

\section{Inferential Statistics}\label{inferential-statistics}

Inferential statistics allow us to make predictions or inferences about
a population based on a sample.

\begin{enumerate}
\def\labelenumi{\arabic{enumi}.}
\tightlist
\item
  \textbf{Hypothesis Testing:} A method to test assumptions regarding
  population parameters using sample data.
\item
  \textbf{Confidence Intervals:} Define a range of values derived from
  sample statistics that likely encompass the true population parameter.
\end{enumerate}

\section{Practical R Commands and
Functions}\label{practical-r-commands-and-functions}

Understanding and utilizing R functions is crucial for effective data
analysis. Some key functions include:

\begin{itemize}
\tightlist
\item
  \texttt{mean()}: Calculates the average.
\item
  \texttt{sd()}: Computes standard deviation.
\item
  \texttt{t.test()}: Performs a t-test for hypothesis testing.
\end{itemize}

\bookmarksetup{startatroot}

\chapter{Visualizing Data with
Graphs}\label{visualizing-data-with-graphs}

\section{Significance of Data
Visualization}\label{significance-of-data-visualization}

Visualization enhances comprehension by allowing researchers to observe
patterns, trends, and anomalies effectively.

\section{Types of Graphs}\label{types-of-graphs}

Variety in graph types caters to different data presentation needs:

\begin{longtable}[]{@{}
  >{\raggedright\arraybackslash}p{(\linewidth - 2\tabcolsep) * \real{0.2963}}
  >{\raggedright\arraybackslash}p{(\linewidth - 2\tabcolsep) * \real{0.7037}}@{}}
\toprule\noalign{}
\begin{minipage}[b]{\linewidth}\raggedright
Graph Type
\end{minipage} & \begin{minipage}[b]{\linewidth}\raggedright
Use Case
\end{minipage} \\
\midrule\noalign{}
\endhead
\bottomrule\noalign{}
\endlastfoot
Bar Graph & Comparing categorical data \\
Histogram & Displaying distribution of continuous data \\
Box Plot & Summarizing data distributions and spotting outliers \\
Scatter Plot & Investigating relationships between two quantitative
variables \\
\end{longtable}

\section{Implementing Visualization in
RKWard}\label{implementing-visualization-in-rkward}

Students will learn how to create visualizations within RKWard by:

\begin{enumerate}
\def\labelenumi{\arabic{enumi}.}
\tightlist
\item
  Navigating to the graph creation menu.
\item
  Selecting the appropriate kind of graph.
\item
  Customizing aesthetics (titles, colors, axes).
\item
  Producing and exporting diagrams for reports.
\end{enumerate}

\bookmarksetup{startatroot}

\chapter{Practical Applications of
Statistics}\label{practical-applications-of-statistics}

\section{Case Studies in Various
Fields}\label{case-studies-in-various-fields}

Statistics plays a pivotal role in diverse disciplines:

\begin{longtable}[]{@{}
  >{\raggedright\arraybackslash}p{(\linewidth - 2\tabcolsep) * \real{0.2424}}
  >{\raggedright\arraybackslash}p{(\linewidth - 2\tabcolsep) * \real{0.7576}}@{}}
\toprule\noalign{}
\begin{minipage}[b]{\linewidth}\raggedright
Field
\end{minipage} & \begin{minipage}[b]{\linewidth}\raggedright
Application
\end{minipage} \\
\midrule\noalign{}
\endhead
\bottomrule\noalign{}
\endlastfoot
Healthcare & Analyzing medical test results, outcomes of treatments, and
patient demographics \\
Business & Applied for market analyses, customer satisfaction studies,
and financial forecasting \\
Social Sciences & Employed in surveys to understand populations,
opinions, and behavioral patterns \\
\end{longtable}

\section{Utilizing Statistical Methods for Decision
Making}\label{utilizing-statistical-methods-for-decision-making}

\begin{itemize}
\tightlist
\item
  Using statistical evidence to guide business strategies.
\item
  Making informed policy decisions based on empirical data.
\item
  Reporting results in a way that is understandable to stakeholders,
  ensuring transparency.
\end{itemize}

\bookmarksetup{startatroot}

\chapter{Summary}\label{summary}

The ``Basic Statistics Using GUI-R (RK Ward)'' course equips learners
with the foundational and practical skills needed for statistical
analysis using R. Students will understand theoretical concepts, grasp
practical applications, and use RKWard effectively to analyze real-world
data.

\section{Key Takeaways}\label{key-takeaways}

\begin{itemize}
\tightlist
\item
  Proficiency in defining and using variables and data types.
\item
  Capability to import and manipulate data in RKWard.
\item
  Understanding of basic statistical practices and their applications.
\item
  Skill in visualizing data for effective communication of results.
\end{itemize}

\bookmarksetup{startatroot}

\chapter{References}\label{references-1}

\begin{enumerate}
\def\labelenumi{\arabic{enumi}.}
\tightlist
\item
  Mohanty, B., \& Misra, S. (2016). \emph{Statistics for Behavioral and
  Social Sciences}.
\item
  Pandya, K., Joshi, P., Bulsari, S., \& Nachane, D. M. (2018).
  \emph{Statistical Analysis in Simple Steps Using R}.
\item
  Field, A. P., Miles, J., \& Field, Z. (2012). \emph{Discovering
  Statistics Using R}.
\item
  Harris, J. K. (2019). \emph{Statistics with R: Solving Problems Using
  Real-World Data}. SAGE Publications.
\item
  Kuhnert, P., \& Venables, B. (2013). \emph{An Introduction to R:
  Software for Statistical Modeling \& Computing}.
\item
  Maindonald, J. H. (2010). \emph{Using R for Data Analysis and
  Graphics}.
\end{enumerate}

\begin{center}\rule{0.5\linewidth}{0.5pt}\end{center}

\emph{This detailed eBook aggregates insights from lectures and
presentation materials, offering an in-depth guide to mastering
statistics using R, specifically through RKWard. It blends theory with
practice to ensure a thorough understanding of both rudimentary and
advanced statistical techniques and includes essential tables for better
comprehension.}

\bookmarksetup{startatroot}

\chapter{- basic-statistics\_2}\label{basic-statistics_2}

\bookmarksetup{startatroot}

\chapter{Introduction}\label{introduction-3}

\section{Purpose of the eBook}\label{purpose-of-the-ebook}

This eBook aims to provide a comprehensive understanding of basic
statistics, focusing on the essential principles necessary for data
analysis\ldots{}

\section{Importance of Statistics}\label{importance-of-statistics}

Statistics is critical in interpreting data efficiently and
effectively\ldots{}

\bookmarksetup{startatroot}

\chapter{Basic Concepts of
Statistics}\label{basic-concepts-of-statistics}

\section{Overview of Statistics}\label{overview-of-statistics}

Statistics is the discipline that deals with the collection,
analysis\ldots{}

\section{Types of Data}\label{types-of-data}

\begin{itemize}
\tightlist
\item
  \textbf{Qualitative Data}: Represents categories\ldots{}
\item
  \textbf{Quantitative Data}:

  \begin{itemize}
  \tightlist
  \item
    \textbf{Discrete Data}: Countable values\ldots{}
  \item
    \textbf{Continuous Data}: Measurable values\ldots{}
  \end{itemize}
\end{itemize}

\section{Descriptive vs.~Inferential
Statistics}\label{descriptive-vs.-inferential-statistics}

\begin{itemize}
\tightlist
\item
  \textbf{Descriptive Statistics}\ldots{}
\item
  \textbf{Inferential Statistics}\ldots{}
\end{itemize}

\bookmarksetup{startatroot}

\chapter{Measures of Central
Tendency}\label{measures-of-central-tendency}

\section{Definition and Importance}\label{definition-and-importance}

Measures of central tendency\ldots{}

\section{The Mean}\label{the-mean}

The mean is the arithmetic average\ldots{}

\subsection{Example with Explanation}\label{example-with-explanation}

Consider the data: 2, 3, 5, 7, 11\ldots{}

\section{The Median}\label{the-median}

The median is the middle value\ldots{}

\subsection{Example with Explanation}\label{example-with-explanation-1}

Consider the data: 3, 5, 1, 7, 9\ldots{}

\section{The Mode}\label{the-mode}

The mode is the value that appears most\ldots{}

\subsection{Example with Explanation}\label{example-with-explanation-2}

Given the dataset: 2, 4, 4, 5, 5, 5, 7, 8\ldots{}

\section{Comparison of Measures}\label{comparison-of-measures}

\begin{longtable}[]{@{}
  >{\raggedright\arraybackslash}p{(\linewidth - 6\tabcolsep) * \real{0.1778}}
  >{\raggedright\arraybackslash}p{(\linewidth - 6\tabcolsep) * \real{0.2889}}
  >{\raggedright\arraybackslash}p{(\linewidth - 6\tabcolsep) * \real{0.2444}}
  >{\raggedright\arraybackslash}p{(\linewidth - 6\tabcolsep) * \real{0.2889}}@{}}
\toprule\noalign{}
\begin{minipage}[b]{\linewidth}\raggedright
Measure
\end{minipage} & \begin{minipage}[b]{\linewidth}\raggedright
Description
\end{minipage} & \begin{minipage}[b]{\linewidth}\raggedright
Strengths
\end{minipage} & \begin{minipage}[b]{\linewidth}\raggedright
Limitations
\end{minipage} \\
\midrule\noalign{}
\endhead
\bottomrule\noalign{}
\endlastfoot
Mean & Average\ldots{} & Utilizes all data points & Sensitive to
outliers \\
Median & Middle\ldots{} & Robust against outliers & Doesn't use all
values \\
Mode & Most frequent & Useful for categorical & May not exist or be
unique \\
\end{longtable}

\bookmarksetup{startatroot}

\chapter{Measures of Variability}\label{measures-of-variability}

\section{Definition and Importance}\label{definition-and-importance-1}

Measures of variability (or dispersion)\ldots{}

\section{Range}\label{range}

The range is the difference\ldots{}

\subsection{Example}\label{example}

Data: 4, 8, 2, 10, 6\ldots{}

\section{Variance}\label{variance}

Variance is the average\ldots{}

\subsection{Example}\label{example-1}

Data: 2, 4, 4, 4, 5, 5, 7\ldots{}

\section{Standard Deviation}\label{standard-deviation}

Standard Deviation is the square root\ldots{}

\section{Interquartile Range (IQR)}\label{interquartile-range-iqr}

\subsection{Example}\label{example-2}

Data: 1, 2, 3, 4, 5, 6, 7, 8, 9\ldots{}

\bookmarksetup{startatroot}

\chapter{Probability Fundamentals}\label{probability-fundamentals}

\section{Introduction to Probability}\label{introduction-to-probability}

Probability measures the likelihood\ldots{}

\section{Types of Events}\label{types-of-events}

\begin{itemize}
\tightlist
\item
  \textbf{Independent Events}
\item
  \textbf{Dependent Events}
\item
  \textbf{Mutually Exclusive Events}
\end{itemize}

\section{Basic Probability Rules}\label{basic-probability-rules}

\begin{enumerate}
\def\labelenumi{\arabic{enumi}.}
\tightlist
\item
  \textbf{Addition Rule}: \[
  P(A \cup B) = P(A) + P(B)
  \]
\item
  \textbf{Multiplication Rule}: \[
  P(A \cap B) = P(A) \times P(B)
  \]
\end{enumerate}

\section{Introduction to Probability
Distributions}\label{introduction-to-probability-distributions}

\subsection{Normal Distribution}\label{normal-distribution}

\begin{itemize}
\tightlist
\item
  Symmetric about the mean\ldots{}
\item
  Properties: Mean = Median = Mode\ldots{}
\end{itemize}

\bookmarksetup{startatroot}

\chapter{Detailed Transcripts}\label{detailed-transcripts}

\section{Transcript from Lec06}\label{transcript-from-lec06}

\textbf{Key Discussion Points}: - Effects of outliers on mean\ldots{} -
Mean properties\ldots{}

\section{Transcript from Lec07}\label{transcript-from-lec07}

\textbf{Key Discussion Points}: - Range, variance, SD\ldots{}

\section{Transcript from Lec08}\label{transcript-from-lec08}

\textbf{Key Discussion Points}: - Z score explanation\ldots{} - Galton
board illustration\ldots{}

\section{Transcript from Lec09}\label{transcript-from-lec09}

\textbf{Key Discussion Points}: - Intro to distributions\ldots{} -
Probability terms\ldots{}

\bookmarksetup{startatroot}

\chapter{Summary of Week 2 Content}\label{summary-of-week-2-content}

\begin{itemize}
\tightlist
\item
  Measures of central tendency
\item
  Measures of variability
\item
  Basics of probability\ldots{}
\end{itemize}

\bookmarksetup{startatroot}

\chapter{Tables and Visualizations}\label{tables-and-visualizations}

\section{Frequency Distribution
Example}\label{frequency-distribution-example}

\begin{longtable}[]{@{}ll@{}}
\toprule\noalign{}
Value & Frequency \\
\midrule\noalign{}
\endhead
\bottomrule\noalign{}
\endlastfoot
1 & 4 \\
2 & 6 \\
3 & 3 \\
4 & 2 \\
5 & 1 \\
\end{longtable}

\section{Interquartile Range Example}\label{interquartile-range-example}

\begin{longtable}[]{@{}ll@{}}
\toprule\noalign{}
Position & Value \\
\midrule\noalign{}
\endhead
\bottomrule\noalign{}
\endlastfoot
1 & 12 \\
2 & 30 \\
3 & 45 \\
4 & 57 \\
5 & 70 \\
\end{longtable}

\[
\text{IQR} = 57 - 30 = 27
\]

\section{Box Plot Visualization}\label{box-plot-visualization}

A box plot visualizes:

\begin{itemize}
\tightlist
\item
  Minimum
\item
  \(Q1\)
\item
  Median
\item
  \(Q3\)
\item
  Maximum
\end{itemize}

\bookmarksetup{startatroot}

\chapter{References}\label{references-2}

\phantomsection\label{refs}
\begin{CSLReferences}{0}{1}
\begin{enumerate}
\def\labelenumi{\arabic{enumi}.}
\tightlist
\item
  Mohanty, B., \& Misra, S. (2016). \emph{Statistics for Behavioral and
  Social Sciences}.
\item
  Pandya, K., et al.~(2018). \emph{Statistical Analysis in Simple Steps
  Using R}.
\item
  Field, A. P., et al.~(2012). \emph{Discovering Statistics Using R}.
\item
  Harris, J. K. (2019). \emph{Statistics with R: Solving Problems Using
  Real-World Data}.
\item
  Anderson, D. R., et al.~(2016). \emph{Statistics for Business \&
  Economics}.
\end{enumerate}

\end{CSLReferences}

\bookmarksetup{startatroot}

\chapter{Appendices}\label{appendices}

\begin{itemize}
\tightlist
\item
  Additional exercises
\item
  Data sets
\item
  Online resources on RKWard
\end{itemize}




\end{document}

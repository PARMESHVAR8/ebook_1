% Options for packages loaded elsewhere
% Options for packages loaded elsewhere
\PassOptionsToPackage{unicode}{hyperref}
\PassOptionsToPackage{hyphens}{url}
\PassOptionsToPackage{dvipsnames,svgnames,x11names}{xcolor}
%
\documentclass[
  letterpaper,
  DIV=11,
  numbers=noendperiod]{scrreprt}
\usepackage{xcolor}
\usepackage[margin=1in]{geometry}
\usepackage{amsmath,amssymb}
\setcounter{secnumdepth}{5}
\usepackage{iftex}
\ifPDFTeX
  \usepackage[T1]{fontenc}
  \usepackage[utf8]{inputenc}
  \usepackage{textcomp} % provide euro and other symbols
\else % if luatex or xetex
  \usepackage{unicode-math} % this also loads fontspec
  \defaultfontfeatures{Scale=MatchLowercase}
  \defaultfontfeatures[\rmfamily]{Ligatures=TeX,Scale=1}
\fi
\usepackage{lmodern}
\ifPDFTeX\else
  % xetex/luatex font selection
\fi
% Use upquote if available, for straight quotes in verbatim environments
\IfFileExists{upquote.sty}{\usepackage{upquote}}{}
\IfFileExists{microtype.sty}{% use microtype if available
  \usepackage[]{microtype}
  \UseMicrotypeSet[protrusion]{basicmath} % disable protrusion for tt fonts
}{}
\makeatletter
\@ifundefined{KOMAClassName}{% if non-KOMA class
  \IfFileExists{parskip.sty}{%
    \usepackage{parskip}
  }{% else
    \setlength{\parindent}{0pt}
    \setlength{\parskip}{6pt plus 2pt minus 1pt}}
}{% if KOMA class
  \KOMAoptions{parskip=half}}
\makeatother
% Make \paragraph and \subparagraph free-standing
\makeatletter
\ifx\paragraph\undefined\else
  \let\oldparagraph\paragraph
  \renewcommand{\paragraph}{
    \@ifstar
      \xxxParagraphStar
      \xxxParagraphNoStar
  }
  \newcommand{\xxxParagraphStar}[1]{\oldparagraph*{#1}\mbox{}}
  \newcommand{\xxxParagraphNoStar}[1]{\oldparagraph{#1}\mbox{}}
\fi
\ifx\subparagraph\undefined\else
  \let\oldsubparagraph\subparagraph
  \renewcommand{\subparagraph}{
    \@ifstar
      \xxxSubParagraphStar
      \xxxSubParagraphNoStar
  }
  \newcommand{\xxxSubParagraphStar}[1]{\oldsubparagraph*{#1}\mbox{}}
  \newcommand{\xxxSubParagraphNoStar}[1]{\oldsubparagraph{#1}\mbox{}}
\fi
\makeatother


\usepackage{longtable,booktabs,array}
\usepackage{calc} % for calculating minipage widths
% Correct order of tables after \paragraph or \subparagraph
\usepackage{etoolbox}
\makeatletter
\patchcmd\longtable{\par}{\if@noskipsec\mbox{}\fi\par}{}{}
\makeatother
% Allow footnotes in longtable head/foot
\IfFileExists{footnotehyper.sty}{\usepackage{footnotehyper}}{\usepackage{footnote}}
\makesavenoteenv{longtable}
\usepackage{graphicx}
\makeatletter
\newsavebox\pandoc@box
\newcommand*\pandocbounded[1]{% scales image to fit in text height/width
  \sbox\pandoc@box{#1}%
  \Gscale@div\@tempa{\textheight}{\dimexpr\ht\pandoc@box+\dp\pandoc@box\relax}%
  \Gscale@div\@tempb{\linewidth}{\wd\pandoc@box}%
  \ifdim\@tempb\p@<\@tempa\p@\let\@tempa\@tempb\fi% select the smaller of both
  \ifdim\@tempa\p@<\p@\scalebox{\@tempa}{\usebox\pandoc@box}%
  \else\usebox{\pandoc@box}%
  \fi%
}
% Set default figure placement to htbp
\def\fps@figure{htbp}
\makeatother





\setlength{\emergencystretch}{3em} % prevent overfull lines

\providecommand{\tightlist}{%
  \setlength{\itemsep}{0pt}\setlength{\parskip}{0pt}}



 


\KOMAoption{captions}{tableheading}
\usepackage{enumitem}
\setlist[enumerate]{leftmargin=*,itemsep=0.5em,parsep=0.5em}
\setlist[itemize]{leftmargin=*,itemsep=0.5em,parsep=0.5em}
\setlist[description]{leftmargin=*,itemsep=0.5em,parsep=0.5em}
\setlist[enumerate]{leftmargin=*,itemsep=0.5em}
\setlist[itemize]{leftmargin=*,itemsep=0.5em}
\makeatletter
\@ifpackageloaded{bookmark}{}{\usepackage{bookmark}}
\makeatother
\makeatletter
\@ifpackageloaded{caption}{}{\usepackage{caption}}
\AtBeginDocument{%
\ifdefined\contentsname
  \renewcommand*\contentsname{Table of contents}
\else
  \newcommand\contentsname{Table of contents}
\fi
\ifdefined\listfigurename
  \renewcommand*\listfigurename{List of Figures}
\else
  \newcommand\listfigurename{List of Figures}
\fi
\ifdefined\listtablename
  \renewcommand*\listtablename{List of Tables}
\else
  \newcommand\listtablename{List of Tables}
\fi
\ifdefined\figurename
  \renewcommand*\figurename{Figure}
\else
  \newcommand\figurename{Figure}
\fi
\ifdefined\tablename
  \renewcommand*\tablename{Table}
\else
  \newcommand\tablename{Table}
\fi
}
\@ifpackageloaded{float}{}{\usepackage{float}}
\floatstyle{ruled}
\@ifundefined{c@chapter}{\newfloat{codelisting}{h}{lop}}{\newfloat{codelisting}{h}{lop}[chapter]}
\floatname{codelisting}{Listing}
\newcommand*\listoflistings{\listof{codelisting}{List of Listings}}
\makeatother
\makeatletter
\makeatother
\makeatletter
\@ifpackageloaded{caption}{}{\usepackage{caption}}
\@ifpackageloaded{subcaption}{}{\usepackage{subcaption}}
\makeatother
\usepackage{bookmark}
\IfFileExists{xurl.sty}{\usepackage{xurl}}{} % add URL line breaks if available
\urlstyle{same}
\hypersetup{
  pdftitle={my-ebook},
  pdfauthor={Parmeshvar},
  colorlinks=true,
  linkcolor={blue},
  filecolor={Maroon},
  citecolor={Blue},
  urlcolor={Blue},
  pdfcreator={LaTeX via pandoc}}


\title{my-ebook}
\author{Parmeshvar}
\date{2025-07-06}
\begin{document}
\maketitle

\renewcommand*\contentsname{Table of contents}
{
\hypersetup{linkcolor=}
\setcounter{tocdepth}{2}
\tableofcontents
}

\bookmarksetup{startatroot}

\chapter{Introduction}\label{introduction}

Introduction

\textbf{DR.Harsh Pradhan}, {[}Phone: +91-9930034241 , Email:
harsh.231284@gmail.com{]},
\href{http://www.uni-potsdam.de/en/university-of-potsdam.html}{Institute
of Management Studies, Banaras Hindu University}, Address: 18-GF,
Jaipuria Enclave, Kaushambhi, Ghaziabad, India, 2010\\
\textbf{Interest}:
\href{http://www.uni-potsdam.de/humfak/hum-forschungsschwerpunkte/forschungscluster-sprache.html}{Goal
Orientation Job Performance Consumer Behavior Behavioral Finance
Bibiliometric Analysis Options as Derivatives Statistics Indian
Knowledge System},

\href{https://orcid.org/0000-0002-3332-3610}{Orcid ID}\\
\href{https://scholar.google.com/citations?user=8l5MEd0AAAAJ&hl=en&oi=sra}{Google
Scholar}\\
\href{http://www.youtube.com/@dr.harshpradhan742}{Youtube ID}

\href{https://bhu.ac.in/Site/FacultyProfile/1_5?FA000562}{Academic
Profile}

\begin{center}\rule{0.5\linewidth}{0.5pt}\end{center}

Courses offered:

\begin{enumerate}
\def\labelenumi{\arabic{enumi}.}
\tightlist
\item
  Free online course, four weeks (MOOC), enrollments open: Introduction
  to Bayesian Data Analysis
\item
  Short (four-hour) tutorial on Bayesian statistics, taught at EMLAR
  2022: here
\item
  Introduction to (frequentist) statistics
\item
  Introduction to Bayesian data analysis for cognitive science
\item
  BDA cover
\end{enumerate}

\section{Lecture notes}\label{lecture-notes}

Download from
\href{https://drive.google.com/drive/folders/1H-jQNOqS3PLgdXADRYF7ag-WqN_tK4qc?usp=share_link}{here}.

\section{Moodle website}\label{moodle-website}

All communications with students in Potsdam will be done through
\href{https://moodle2.uni-potsdam.de/course/view.php?id=17526}{this
website}. \# 📘 Schedule

\begin{longtable}[]{@{}
  >{\raggedright\arraybackslash}p{(\linewidth - 10\tabcolsep) * \real{0.0361}}
  >{\raggedright\arraybackslash}p{(\linewidth - 10\tabcolsep) * \real{0.0469}}
  >{\raggedright\arraybackslash}p{(\linewidth - 10\tabcolsep) * \real{0.1047}}
  >{\raggedright\arraybackslash}p{(\linewidth - 10\tabcolsep) * \real{0.1625}}
  >{\raggedright\arraybackslash}p{(\linewidth - 10\tabcolsep) * \real{0.3430}}
  >{\raggedright\arraybackslash}p{(\linewidth - 10\tabcolsep) * \real{0.3069}}@{}}
\toprule\noalign{}
\begin{minipage}[b]{\linewidth}\raggedright
\textbf{Week}
\end{minipage} & \begin{minipage}[b]{\linewidth}\raggedright
\textbf{Lecture}
\end{minipage} & \begin{minipage}[b]{\linewidth}\raggedright
\textbf{Main Topic}
\end{minipage} & \begin{minipage}[b]{\linewidth}\raggedright
\textbf{Subtopic}
\end{minipage} & \begin{minipage}[b]{\linewidth}\raggedright
\textbf{🎥 Video}
\end{minipage} & \begin{minipage}[b]{\linewidth}\raggedright
\textbf{📄 PDF Resource}
\end{minipage} \\
\midrule\noalign{}
\endhead
\bottomrule\noalign{}
\endlastfoot
Week 2 & 1 & Descriptive Statistics & Central Tendency &
\href{https://youtu.be/9709cbHi4-8?t=173}{Video} &
\href{https://drive.google.com/file/d/1CXkLjlvsMt6-BcrDLrNPsTt6SIUSq7NG/view}{Week
2.pdf} \\
& 2 & Descriptive Statistics & Measure of Variability &
\href{https://youtu.be/Wn-buo8ARUY}{Video} & Same as above \\
& 3 & Descriptive Statistics & Describing Data &
\href{https://youtu.be/LS9jxi9t06A}{Video} & Same as above \\
& 4 & Descriptive Statistics & Probability &
\href{https://youtu.be/DImY6pmX8UA}{Video} & Same as above \\
& 5 & Descriptive Statistics & Distribution &
\href{https://youtu.be/RZVFQ_wlQBI}{Video} & Same as above \\
Week 3 & 1 & Descriptive Statistics & Z Table (Normal Distribution) &
\href{https://youtu.be/mh92FQKFFlA}{Video} &
\href{https://drive.google.com/file/d/1sVwyzDjitecinjyR4LxbOafHe8ol0JIu/view}{Week
3.pdf} \\
& 2 & Descriptive Statistics & Measuring Divergence &
\href{https://youtu.be/SdghNy_5oe0}{Video} & Same as above \\
& 3 & Inferential Statistics & Sample and Population &
\href{https://youtu.be/eYjtWoMb21s}{Video} & Same as above \\
& 4 & Inferential Statistics & Model Fit &
\href{https://youtu.be/1hdzMWcj5a0}{Video} & Same as above \\
& 5 & Inferential Statistics & Hypothesis and Error &
\href{https://youtu.be/pP3yKhIgJQw}{Video} & Same as above \\
Week 4 & 1 & Terms of Statistics & Terms of Statistics &
\href{https://youtu.be/lO7jZJIlTgU}{Video} &
\href{https://drive.google.com/file/d/1dDDUXCdCBp6VxKbQzsnsIwQY-kZL8U3D/view}{Week
4.pdf} \\
& 2 & Terms of Statistics & T-Test &
\href{https://youtu.be/4i-2uZ2LuXg}{Video} & Same as above \\
& 3 & Terms of Statistics & T-Test in Detail &
\href{https://youtu.be/Vo48lLCR3yk}{Video} & Same as above \\
& 4 & ANOVA & ANOVA & \href{https://youtu.be/jz-lu_HtkXo}{Video} & Same
as above \\
Week 5 & 1 & ANOVA & Example of ANOVA &
\href{https://youtu.be/AQL2fVlc2EA}{Video} &
\href{https://drive.google.com/file/d/1f3LZJRQS-Lnxy0_kNaAASZppPJk8pUX_/view}{Week
5.pdf} \\
& 2 & ANOVA & Types of ANOVA &
\href{https://youtu.be/MQ9GFyXZtQA}{Video} & Same as above \\
& 3 & Correlation & Introduction to Correlation &
\href{https://youtu.be/5zMCSIa2YqE}{Video} & Same as above \\
& 4 & Correlation & Regression (Part 1) &
\href{https://youtu.be/ELEeZNyySSM}{Video} & Same as above \\
& 5 & Correlation & Regression (Part 2) &
\href{https://youtu.be/3hJdWKV0alw}{Video} & Same as above \\
Week 6 & 1 & Correlation & R Script for Regression &
\href{https://youtu.be/V151yC8cCMs}{Video} &
\href{https://drive.google.com/file/d/1mA1-nFxjmXK_QC8vbW4BhoKO6HNS0wZy/view}{Week
6.pdf} \\
& 2 & Chi Square & Chi Square &
\href{https://youtu.be/Z9DxCZzXAz4}{Video} & Same as above \\
& 3 & Chi Square & Chi Square Test &
\href{https://youtu.be/ZhUTidqrPbI}{Video} & Same as above \\
& 4 & Logistic Function & Regression Function &
\href{https://youtu.be/3EXiuBJYTn4}{Video} & Same as above \\
& 5 & Logistic Function & Distribution &
\href{https://youtu.be/9NhApiDlFrs}{Video} & Same as above \\
Week 7 & 1 & Time Series & Intro to Time Series &
\href{https://youtu.be/t5z9cM-PgoQ}{Video} &
\href{https://drive.google.com/file/d/1ACloSOgbYVlxmqGfUe_mBDvirLJzCZnB/view}{Week
7.pdf} \\
& 2 & Time Series & Conditional Probability &
\href{https://youtu.be/SjaJNBilHcM}{Video} & Same as above \\
& 3 & Time Series & Additional Concepts &
\href{https://youtu.be/pm1SXK0kFpg}{Video} & Same as above \\
& 4 & Time Series & Distribution &
\href{https://youtu.be/0V0UTi5fbPc}{Video} & Same as above \\
& 5 & Time Series & Poisson Distribution &
\href{https://youtu.be/MM-MkuJa_KY}{Video} & Same as above \\
& 6 & Index Numbers & Price \& Quantity Index &
\href{https://youtu.be/YZv8H2PlymQ}{Video} & Same as above \\
& 7 & Decision Environments & Risk/Uncertainty, Bayes, Trees &
\href{https://youtu.be/b0jP5anQiUg}{Video} & Same as above \\
& 8 & Time Series Analysis & Components, Trend, Seasonality &
\href{https://youtu.be/C77jZeYkBBU}{Video} & Same as above \\
& 9 & Time Series Analysis & Least Squares Method &
\href{https://youtu.be/yiJjJaL7z5w}{Video} & Same as above \\
Week 8 & 1 & Effect Size \& Documentation & Package/Library &
\href{https://youtu.be/C77jZeYkBBU}{Video} &
\href{https://drive.google.com/file/d/1dUv8rXI3vhsMkBcYnlStYg-xJpYtCYSF/view}{Week
8.pdf} \\
& 2 & Effect Size \& Documentation & RStudio vs RKward &
\href{https://youtu.be/Z9DxCZzXAz4}{Video} & Same as above \\
& 3 & Effect Size \& Documentation & Flexplot &
\href{https://youtu.be/ZhUTidqrPbI}{Video} & Same as above \\
& 4 & Effect Size \& Documentation & Functions &
\href{https://youtu.be/3EXiuBJYTn4}{Video} & Same as above \\
& 5 & Effect Size \& Documentation & R Shiny \& R Markdown &
\href{https://youtu.be/9NhApiDlFrs}{Video} & Same as above \\
& 6 & Effect Size \& Documentation & Application with Real Datasets &
\href{https://youtu.be/t5z9cM-PgoQ}{Video} & Same as above \\
& 7 & Effect Size \& Interpretation & Importance in Testing &
\href{https://youtu.be/SjaJNBilHcM}{Video} & Same as above \\
& 8 & Effect Size \& Interpretation & Installing dplyr, ggplot2 &
\href{https://youtu.be/MM-MkuJa_KY}{Video} & Same as above \\
& 9 & Effect Size \& Interpretation & Visual Model Interpretation &
\href{https://youtu.be/YZv8H2PlymQ}{Video} & Same as above \\
& 10 & Effect Size \& Interpretation & Creating/Using Functions &
\href{https://youtu.be/b0jP5anQiUg}{Video} & Same as above \\
& 11 & Effect Size \& Interpretation & Report, Dashboard, Interactivity
& \href{https://youtu.be/C77jZeYkBBU}{Video} & Same as above \\
\end{longtable}

\bookmarksetup{startatroot}

\chapter{week 6}\label{week-6}

Table of Contents

Introduction\_

Chi-Square Test of Goodness of Fit

Chi-Square Test of Independence

Non-Parametric Tests

Non-Linear and Logistic Regression

Poisson \& Negative Binomial Distribution

Robust and Bayesian Regression

Model Fit Diagnostics

Exercises, Simulations, \& Datasets

Summary

References

\begin{enumerate}
\def\labelenumi{\arabic{enumi}.}
\tightlist
\item
  Introduction
\end{enumerate}

This Week 6 eBook focuses on advanced statistical procedures for
analyzing categorical and non-normal data using RKWard, a GUI-based
frontend to R.

We address: - When traditional parametric methods fail - Tools for
ordinal, non-linear, or count data - How to interpret diagnostic plots,
residuals, and goodness-of-fit metrics

\begin{enumerate}
\def\labelenumi{\arabic{enumi}.}
\setcounter{enumi}{1}
\tightlist
\item
  Chi-Square Test of Goodness of Fit
\end{enumerate}

Theory Refresher

Use this test to see if observed frequency data matches a theoretical
distribution (e.g., uniform, binomial, Poisson).

📊 Example 1: Dice Fairness

obs \textless- c(9, 7, 6, 4, 5, 5) expected \textless- rep(sum(obs)/6,
6) chisq.test(obs, p = rep(1/6, 6)) 📊 Example 2: Simulated Biased Die
(Monte Carlo)

set.seed(42) sim\_data \textless- sample(1:6, size = 600, replace =
TRUE, prob = c(0.1, 0.1, 0.2, 0.2, 0.2, 0.2)) table\_sim \textless-
table(sim\_data) chisq.test(table\_sim, p = rep(1/6, 6)) 📊 Example 3:
Poisson-GOF for Counts

library(MASS) data\_counts \textless- rpois(100, lambda = 3) obs\_table
\textless- table(data\_counts) exp\_probs \textless-
dpois(as.numeric(names(obs\_table)), lambda = 3) chisq.test(obs\_table,
p = exp\_probs/sum(exp\_probs)) 🎨 Visualizing Frequencies

barplot(rbind(obs, expected), beside = TRUE, col = c(``skyblue'',
``orange''), legend.text = c(``Observed'', ``Expected''), main = ``Dice
Roll Distribution'') 3. Chi-Square Test of Independence 🔍 Purpose Test
whether two categorical variables are independent.

📊 Example 1: Gender vs Preference

df \textless- data.frame( Gender = c(``Male'', ``Male'', ``Female'',
``Female''), Laptop = c(``Gaming'', ``Non-Gaming'', ``Gaming'',
``Non-Gaming''), Freq = c(27, 8, 5, 7) ) table\_df \textless- xtabs(Freq
\textasciitilde{} Gender + Laptop, data = df) chisq.test(table\_df) 📊
Example 2: Titanic Survival

library(datasets) data(Titanic) chisq.test(Titanic) 📊 Example 3:
Simulated Survey

set.seed(123) survey \textless- data.frame( Smoke = sample(c(``Yes'',
``No''), 100, replace = TRUE), Exer = sample(c(``None'', ``Some'',
``Regular''), 100, replace = TRUE) ) tb \textless-
table(survey\(Smoke, survey\)Exer) chisq.test(tb) 📉 Association
Strength

library(vcd) assocstats(tb) 4. Non-Parametric Tests 🎯 Why Use Them?
Parametric assumptions (normality, equal variance) are not always met.
Non-parametric tests allow analysis without these constraints.

📋 Common Tests Parametric Non-Parametric Equivalent One-sample t-test
Wilcoxon Signed-Rank Test Two-sample t-test Mann-Whitney U Test One-Way
ANOVA Kruskal-Wallis Test Two-Way ANOVA Friedman Test Pearson
Correlation Spearman Rank Correlation

📊 Example 1: Wilcoxon Test (Single Sample)

data \textless- c(3.1, 3.6, 3.8, 4.0, 3.5) wilcox.test(data, mu = 3.5)
📊 Example 2: Mann-Whitney (Between Groups)

group\_a \textless- c(10, 12, 14, 16) group\_b \textless- c(8, 9, 10,
11) wilcox.test(group\_a, group\_b) 📊 Example 3: Kruskal-Wallis on Iris

kruskal.test(Sepal.Length \textasciitilde{} Species, data = iris) 📊
Example 4: Spearman Rank Correlation

cor.test(iris\(Sepal.Length, iris\)Petal.Length, method = ``spearman'')
🚧 Next: Part 2 --- covering:

Non-Linear Regression

Logistic Regression

Poisson \& Negative Binomial

Robust \& Bayesian Regression

Model Fit Diagnostics

Simulations, Interactive Plots

\begin{enumerate}
\def\labelenumi{\arabic{enumi}.}
\setcounter{enumi}{4}
\tightlist
\item
  Non-Linear and Logistic Regression
\end{enumerate}

5.1 Non-Linear Regression

Used when data shows curvature, not a straight-line relationship.

📊 Example 1: Quadratic Fit

```r x \textless- 1:10 y \textless- 5 + 2 * x\^{}2 + rnorm(10, 0, 10)
model\_quad \textless- lm(y \textasciitilde{} poly(x, 2, raw = TRUE))
summary(model\_quad) plot(x, y) lines(x, predict(model\_quad), col =
``red'') 📊 Example 2: Exponential Growth

x \textless- 1:20 y \textless- 2 * exp(0.3 * x) + rnorm(20, 0, 10) df
\textless- data.frame(x, y) model\_exp \textless- nls(y
\textasciitilde{} a * exp(b * x), data = df, start = list(a = 1, b =
0.1)) summary(model\_exp) 5.2 Logistic Regression 📊 Example: Student
Pass/Fail

students \textless- data.frame( Hours = c(1,2,3,4,5,6,7,8,9,10), Pass =
c(0,0,0,1,1,1,1,1,1,1) )

log\_model \textless- glm(Pass \textasciitilde{} Hours, data = students,
family = binomial()) summary(log\_model) 🔁 Predict Probabilities

students\(prob <- predict(log_model, type = "response")
plot(students\)Hours, students\$prob, type = ``b'', col = ``blue'') 🎯
ROC Curve

library(pROC) roc\_obj \textless- roc(students\(Pass, students\)prob)
plot(roc\_obj) auc(roc\_obj) 6. Poisson \& Negative Binomial
Distribution 6.1 Poisson: Modeling Rare Events

\{r\}

🔍 Test Fit

observed \textless- table(data\_pois) expected \textless-
dpois(as.numeric(names(observed)), lambda = lambda) chisq.test(observed,
p = expected / sum(expected)) 6.2 Negative Binomial: Handling
Overdispersion

library(MASS) nb\_data \textless- rnbinom(100, size = 5, mu = 4)
hist(nb\_data, col = ``darkred'', main = ``Negative Binomial'') 🔬
Compare Fit

mean(data\_pois); var(data\_pois) \# Poisson: mean ≈ variance
mean(nb\_data); var(nb\_data) \# NB: var \textgreater{} mean 7. Robust
and Bayesian Regression 7.1 Robust Regression

library(MASS) x \textless- 1:10 y \textless- 2*x + rnorm(10) y{[}10{]}
\textless- 100 \# Outlier

model\_rlm \textless- rlm(y \textasciitilde{} x) summary(model\_rlm)
plot(x, y) abline(model\_rlm, col = ``red'') 7.2 Bayesian Regression
(brms)

library(brms) data \textless- data.frame(x = rnorm(100), y = rnorm(100))
model\_brm \textless- brm(y \textasciitilde{} x, data = data, family =
gaussian(), chains = 2, iter = 1000) summary(model\_brm)
plot(model\_brm) 8. Model Fit Diagnostics 🔎 AIC \& BIC

AIC(model\_quad, log\_model) BIC(model\_quad, log\_model) 📈 Residual
Plots

par(mfrow=c(2,2)) plot(log\_model) 🧪 Durbin-Watson Test

library(car) durbinWatsonTest(log\_model) 9. Exercises, Simulations, \&
Datasets 🧠 Challenge 1: Titanic Chi-Square

chisq.test(Titanic) 🧠 Challenge 2: Spearman on mtcars

cor.test(mtcars\(mpg, mtcars\)hp, method = ``spearman'') 🧠 Challenge 3:
Logistic + Polynomial

mtcars\(am <- as.factor(mtcars\)am) log\_mod \textless- glm(am
\textasciitilde{} poly(mpg, 2), data = mtcars, family = binomial())
summary(log\_mod) 🧠 Challenge 4: Negative Binomial Fit

library(MASS) data \textless- rnegbin(100, theta = 2) fit\_nb \textless-
glm.nb(data \textasciitilde{} 1) summary(fit\_nb) 10. Summary This
module brought together:

💡 Chi-Square Tests for independence and fit

🧱 Non-parametric alternatives to parametric tests

🔁 Logistic Regression for classification

📊 Poisson and NB distributions for count data

🧠 Robust and Bayesian inference for resistant modeling

🧪 Diagnostics to ensure model quality

References

Dr.~Harsh Pradhan, BHU Lecture Notes R Core Team (2024). The R Project
for Statistical Computing. MASS, brms, car, vcd, performance, tidyverse
packages Text: Field, A. (2013). Discovering Statistics Using R

🚀 Next Steps

Coming in Part 3:

Multinomial and ordinal logistic regression

Zero-inflated Poisson (ZIP) and hurdle models

Bootstrapping and permutation tests

RMarkdown interactivity: sliders, code widgets

Custom diagnostic dashboards

Expanded regression use cases: finance, healthcare, social science

Brute-force simulations, grid search tuning, multiple datasets

Data cleaning + wrangling using dplyr, janitor, and tidymodels

\begin{enumerate}
\def\labelenumi{\arabic{enumi}.}
\setcounter{enumi}{11}
\tightlist
\item
  Advanced Logistic Models
\end{enumerate}

12.1 Multinomial Logistic Regression

Used when the outcome variable has more than two categories (e.g.,
``Low'', ``Medium'', ``High'').

library(nnet) data(iris) iris\(Size <- cut(iris\)Sepal.Length, breaks=3,
labels=c(``Short'', ``Medium'', ``Long'')) model\_multi \textless-
multinom(Size \textasciitilde{} Sepal.Width + Petal.Length, data=iris)
summary(model\_multi) 12.2 Ordinal Logistic Regression For ordered
categories.

library(MASS) housing \textless- data.frame( Sat = factor(sample(1:3,
100, replace = TRUE), labels = c(``Low'', ``Med'', ``High'')), Infl =
sample(1:5, 100, replace = TRUE), Type = sample(c(``Tower'',
``Apartment'', ``House''), 100, replace = TRUE) ) model\_ord \textless-
polr(Sat \textasciitilde{} Infl + Type, data = housing, Hess=TRUE)
summary(model\_ord) 13. Zero-Inflated and Hurdle Models 13.1
Zero-Inflated Poisson (ZIP) Used when count data has excess zeros.

library(pscl) data(``bioChemists'', package = ``pscl'') zip\_model
\textless- zeroinfl(art \textasciitilde{} fem + mar + kid5 + phd + ment,
data = bioChemists, dist = ``poisson'') summary(zip\_model) 13.2 Hurdle
Model

hurdle\_model \textless- hurdle(art \textasciitilde{} fem + mar + kid5 +
phd + ment, data = bioChemists) summary(hurdle\_model) 14. Bootstrapping
\& Permutation Testing 14.1 Bootstrapping a Mean

library(boot) data \textless- rnorm(50, mean = 10, sd = 3)

mean\_fn \textless- function(data, indices) \{ d \textless-
data{[}indices{]} return(mean(d)) \}

boot\_out \textless- boot(data = data, statistic = mean\_fn, R = 1000)
boot.ci(boot\_out, type = ``bca'') 14.2 Permutation Test Example

set.seed(100) group1 \textless- rnorm(20, mean = 50) group2 \textless-
rnorm(20, mean = 55)

obs\_diff \textless- mean(group1) - mean(group2)

combined \textless- c(group1, group2) perm\_diffs \textless-
replicate(5000, \{ shuffled \textless- sample(combined)
mean(shuffled{[}1:20{]}) - mean(shuffled{[}21:40{]}) \})

p\_value \textless- mean(abs(perm\_diffs) \textgreater= abs(obs\_diff))
hist(perm\_diffs, main = ``Permutation Test'', col = ``lightblue'')
abline(v = obs\_diff, col = ``red'') 15. Interactive Widgets with Quarto
Sliders

sliderInput(``lambda'', ``Poisson Rate:'', min = 1, max = 10, value = 3)
plotOutput(``poisPlot'')

output\(poisPlot <- renderPlot({
  barplot(dpois(0:10, input\)lambda), names.arg = 0:10, main =
paste(``Poisson Distribution with λ ='', input\$lambda), col =
``steelblue'') \}) 16. Data Wrangling Pipelines Cleaning \& Summarizing

library(dplyr) library(janitor)

cleaned \textless- iris \%\textgreater\% clean\_names() \%\textgreater\%
group\_by(species) \%\textgreater\% summarise(across(everything(), mean,
.names = ``avg\_\{.col\}'')) 17. Visual Diagnostics 17.1 Residual
Diagnostics

library(performance) model \textless- lm(mpg \textasciitilde{} wt + hp,
data = mtcars) performance::check\_model(model) 17.2 Leverage \&
Influence

influence.measures(model) plot(hatvalues(model), main = ``Leverage
Values'') 18. Grid Search and Cross Validation Using caret package

library(caret) data(iris)

train\_control \textless- trainControl(method = ``cv'', number = 5) grid
\textless- expand.grid(.k = seq(3, 15, by = 2))

model\_knn \textless- train(Species \textasciitilde{} ., data = iris,
method = ``knn'', trControl = train\_control, tuneGrid = grid)
plot(model\_knn) 19. Case Study: Healthcare Outcomes Predicting hospital
readmission using logistic regression.

set.seed(42) df \textless- data.frame( age = sample(20:90, 200, replace
= TRUE), diabetes = sample(c(0,1), 200, replace = TRUE), readmit =
sample(c(0,1), 200, replace = TRUE) )

logit \textless- glm(readmit \textasciitilde{} age + diabetes, data =
df, family = binomial()) summary(logit) Plot Prediction

df\(pred <- predict(logit, type = "response")
plot(df\)age, df\(pred, col = df\)diabetes + 1, pch = 19, xlab =
``Age'', ylab = ``Predicted Probability'') 20. Massive Simulation:
Chi-Square Distribution

set.seed(123) sim\_data \textless- replicate(10000, \{ obs \textless-
rpois(6, lambda = 10) exp \textless- rep(mean(obs), 6) sum((obs -
exp)\^{}2 / exp) \})

hist(sim\_data, breaks = 50, col = ``gray'', main = ``Chi-Square
Simulated Distribution'') abline(v = qchisq(0.95, df = 5), col =
``red'') 21. Resources for Practice Datasets:

mtcars, iris, Titanic, bioChemists, airquality, faithful

Visual tools:

plotly, ggplot2, performance, brms

Core Packages:

caret, pscl, nnet, MASS, boot, dplyr, tidymodels, vcd

Final Thoughts

Testing relationships (Chi-Square)

Modeling categories (Logistic, Ordinal, Multinomial)

Working with counts (Poisson, ZIP, NB)

Handling noise and outliers (Robust Regression)

Going Bayesian (brms + Stan)

Validating rigorously (cross-validation, bootstrap, ROC, AIC/BIC)

This eBook can be extended to predictive modeling, real-world
dashboards, and reproducible research.

\begin{enumerate}
\def\labelenumi{\arabic{enumi}.}
\setcounter{enumi}{22}
\tightlist
\item
  Project Template: Real-World Case Study Framework
\end{enumerate}

Objective

Develop an end-to-end statistical analysis pipeline using tools covered
in this course.

📁 Dataset: Custom or Open Data Portal

Options: - UCI Machine Learning Repository - Kaggle Datasets - Indian
Government Data Portals (data.gov.in)

Steps:

🔍 Step 1: Problem Definition

Define a question like: \textgreater{} ``Is there an association between
education level and voting preference?''

🧹 Step 2: Data Cleaning

library(tidyverse) data \textless- read.csv(``your\_dataset.csv'')
data\_clean \textless- data \%\textgreater\% janitor::clean\_names()
\%\textgreater\% drop\_na() 📊 Step 3: EDA (Exploratory Data Analysis)

ggplot(data\_clean, aes(x = variable1, fill = factor(variable2))) +
geom\_bar(position = ``dodge'') + theme\_minimal() 📈 Step 4: Modeling
Choose one or more:

Chi-square (for independence)

Logistic Regression (for binary outcomes)

Poisson/NB (for count outcomes)

Non-parametric (when assumptions fail)

🧪 Step 5: Validation

library(performance) check\_model(your\_model) 📋 Step 6: Reporting Use:

Tables

Model summaries

AIC/BIC

Residuals

R² (if applicable)

summary(your\_model) 24. Visual Appendix: Model Diagnostic Gallery
library(performance) library(see)

Example with linear model

model \textless- lm(mpg \textasciitilde{} hp + wt, data = mtcars)

Model diagnostics

check\_model(model) 25. Bonus: Live Simulation Tool with Shiny

Edit library(shiny)

ui \textless- fluidPage( titlePanel(``Poisson Simulator''),
sidebarLayout( sidebarPanel( sliderInput(``lambda'', ``Lambda (Rate)'',
1, 10, value = 3) ), mainPanel( plotOutput(``poisPlot'') ) ) )

server \textless- function(input, output) \{ \# (Poisson barplot code
removed for PDF compatibility) \}

shinyApp(ui = ui, server = server) 26. Advanced Topics for Further
Exploration Topic Package Description Bayesian Multilevel brms, rstan
Hierarchical regression models Structural Equation lavaan Latent
variable modeling Time Series Forecasting forecast, tsibble ARIMA,
exponential smoothing Mixed-Effects Models lme4, nlme Random
intercept/slope models Missing Data Handling mice, missForest Imputation
strategies High-Dimensional Data glmnet Lasso and Ridge regression




\end{document}
